

Consider a situation where the boundary $x=0$ is held at a constant concentration
(for instance one face of a soil or rock sample is exposed to a
tank with fixed concentration of solute). This is a commonly used experimental
method (sometimes referred to as ``diffusion coupon experiment'') for determining
diffusivities in rocks and soils. Suppose the concentration of solute in the
sample is initially zero, and assuming that the sample extends for a
``relatively long'' distance, so that it is practically ``semi-infinite`` along $x$
(see \autoref{fig:coupon-schematic}).

\begin{figure}[hbt!]
\center
\includegraphics[width=0.7\textwidth]{{Cases/diffusionCoupon/coupon-schematic}.png}
\caption{Schematic of the semi-infinite ``diffusion coupon'' experiment.``}
\label{fig:coupon-schematic}
\end{figure}


The diffusion equation and boundary conditions are:

\begin{equation}
    \frac{\partial C}{\partial t} - D \frac{\partial^2 C}{\partial x^2} = 0
    \label{eqn:gov-coupon}
\end{equation}
In a true ``diffusion coupon'', the boundary conditions are:
\begin{equation}
C(0,t) = C_0; C(x,t) \xrightarrow[]{} 0 \textrm{ as } x \xrightarrow[]{} \infty
    % C(0,t) = C_0; C(x,t) \xrightarrow 0 \textrm{ as } x \xrightarrow \infty
\end{equation}
which is generally a good approximation as long as the length of the column is
sufficiently greater than the characteristic length of diffusion.

If this is not the case and a finite domain is needed, the following boundary
conditions can be used in the model:
\begin{equation}
    C(0,t) = C_0; -D \frac{\partial C}{\partial x} = 0 \textrm{ on } x=L
\end{equation}
such that there is a no-flux (2nd-type, Neumann) BC on the boundary $x=L$.
This can be modeled as a finite domain using a summation of fundamental
diffusion equation solutions using image sources to represent the no-flux BC on
the boundary. We therefore compare the pyDiffusionFDM model to results from
both the semi-infinite domain and the finite domain analytical solutions. 

\subsection{Analytical Solutions}

For the semi-infinite domain case, the analytical solution is as follows:
\begin{equation}
    C(x,t) = C_0 \textrm{erfc}\left( \frac{x}{2 \sqrt{Dt}} \right)
\end{equation}

For the finite domain case, the analytical solution can be constructed as follows:
\begin{equation}
C(x,t) =  C_0 - 4 \sum^\infty_{k=1} \left[ \sin \left( 2k+1) \pi \frac{x}{2L} \right) \cdot \exp{ - \frac{(2k+1)^2 \pi^2 Dt}{4L^2 (2k+1) \pi} } \right]
\end{equation}



\subsection{Input File}
\verbatiminput{Cases/diffusionCoupon/inputFile.yml}

\subsection{Results}

\begin{figure}[hbt!]
\center
\includegraphics[width=0.7\textwidth]{{Cases/diffusionCoupon/compare_ana_plot_1}.pdf}
\caption{Comparison of concentration profiles to two analytical solutions, one
with semi-infinite domain and one with a finite domain using a summation of
image sources.}
\end{figure}


\begin{figure}[hbt!]
\center
\includegraphics[width=0.7\textwidth]{{Cases/diffusionCoupon/compare_ana_plot_2}.pdf}
\caption{Comparison of breakthrough curves to two analytical solutions, one
with semi-infinite domain and one with a finite domain using a summation of
image sources.}
\end{figure}

Add more info later...
